
\section{Introduction}

Since computation has entered into the exascale era, the main concern for now is how to minimize the power consumption of the components used. Numerous hardware and software based methodologies have been used or currently are being researched in this regard. GPUs are comparatively newer than the other accelerators but making them energy efficient is one of the main goals for the researchers. Because of having high computational power they are being used  extensively in HPC systems. In fact many of the top most supercomputers are using GPUs along with CPUs to achieve more power in processing and computation. Machines built with CPU-GPUs consume a lot of power but it needs to be mitigated because of green computing. Lighten up the power consumption of GPUs has not reached it's peak yet unlike the CPUs. The terms energy and power are very much related as energy is the product of power and time. Power consumption of a system depends on the center architecture and system goals and this is applicable for all the accelerators. At the same time power wastage should me minimalized.

This paper --- gives insights about the contribuiton of GPUs in HPC along with the current trends. This also highlights some comparisons among the accelerators and some related research works. Potential techniques to achieve energy efficiency for GPUs and possible future works in this regard.